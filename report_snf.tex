\documentclass[10pt]{article}

% Packages and macros
% Common packages
\usepackage[T1]{fontenc}
\usepackage{lmodern}
\usepackage[utf8]{inputenc}
\usepackage{vmargin}
\usepackage{mathrsfs, mathenv}
\usepackage{amsmath, amsthm, amssymb, amsfonts, amscd}
\usepackage{graphicx}
\usepackage{hyperref}
\hypersetup{citecolor=blue, colorlinks=true, linkcolor=black}
\usepackage[capitalise]{cleveref}
\usepackage{autonum}

% bibliography
\usepackage{bibentry}
\nobibliography*

% tables
\usepackage{booktabs}

% plots
\usepackage{pgfplots}
\usepackage{tikz}
\usetikzlibrary{patterns,arrows,decorations.pathmorphing,backgrounds,positioning,fit,matrix}
\usepackage[labelfont=bf]{caption}
\setlength{\belowcaptionskip}{-5pt}
\usepackage{here}
\usepackage[font=normal]{subcaption}

% Prevent itemized lists from running into the left margin inside theorems and proofs
\usepackage{enumitem}
\setlist[itemize]{leftmargin=.5in}
\setlist[enumerate]{leftmargin=.5in,topsep=3pt,itemsep=3pt,label=(\roman*)}

\newcommand{\abs}[1]{\left\lvert#1\right\rvert}
\newcommand{\norm}[1]{\left\|#1\right\|}
\renewcommand{\phi}{\varphi}
\renewcommand{\theta}{\vartheta}
\newcommand{\totext}[1]{\ensuremath{\stackrel{#1}{\to}}}
\newcommand{\R}{\mathbb{R}}
\newcommand{\epl}{\varepsilon}
\newcommand{\defeq}{\coloneqq}
\newcommand{\eqdef}{\eqqcolon}
\newcommand{\E}{\operatorname{\mathbb{E}}}
\newcommand{\Hell}{d_{\mathrm{Hell}}}
\renewcommand{\d}{\mathrm{d}}
\newcommand{\dd}{\,\mathrm{d}}


% Title
\title{Scientific Report on SNSF project 172710 \\ Numerical methods for stationary and time-dependent multiscale problems}
\author{Professor Assyr Abdulle \\
	Chair of Numerical Analysis and Computational Mathematics (ANMC) \\
	MATH, EPFL, Station 8, 1015 Lausanne, Switzerland
}
\date{}

\begin{document}
	
\maketitle	

\section{Summary of the research work and its results}

In this project we have focused on a series of diverse aspects of multiscale differential problems \cite{BLP78,CiD99} and their numerical solution \cite{AEE12}. Multiscale differential equations arise in a wide range of models and find applications in both applied and social sciences. Often, the discretization needed for solving multiscale equations numerically is constrained by the characteristic length of the smallest size in the problem. For this reason, we employ the theory of homogenization, which allows to obtain effective models which give a macroscopic description of the multiscale solution, and in practice to solve numerically the equations. The multiplicity of scales in the problem could be spatial, e.g. in elasticity or heat transfer problems for heterogeneous materials, or temporal, e.g. in modelling chemical reactions with a fast/slow behaviour or in modelling financial markets with a micro/macro structure.

The content of this project can be roughly divided in two parts. The first concerns the numerical solution of multiscale \textit{forward} problems, and is summarized in \cref{sec:Edoardo,sec:Rosilho_1,sec:Rosilho_2}. In this part, we consider the data of the problem (i.e., elasticity of a heterogeneous material, volatility of a financial market, ...) to be given, and design novel analytical and numerical techniques for approximate the solution of the underlying differential equations. In the second part, which is summarized in \cref{sec:AndreaDB,sec:AndreaZ} we instead consider multiscale \textit{inverse} problems, for which the data is the unknown, and a solution is given as observations. The more fundamental topic of \textit{probabilistic numerical methods}, which can be employed to milden some issues in both the forward and the inverse solutions, is finally summarized in \cref{sec:Giacomo}.

\subsection{Corrector problems with enhanced resonance errors in the FE-HMM}\label{sec:Edoardo}

The finite element heterogeneous multiscale method (FE-HMM) \cite{Abd05b,EE03} is a numerical scheme designed for partial differential equations (PDEs) with highly-oscillatory coefficients. In this setting, a fine mesh is needed to approximate the solution with a standard finite element solver. Indeed, the mesh characteristic size cannot exceed the oscillation size if the goal is capturing the microscopic features of the solution. If we are only interested in the macroscopic behaviour, we can instead rely on to the theory of homogenization \cite{BLP78,CiD99} and first compute an effective, or \textit{homogenized}, non-oscillatory coefficient, and then solve numerically the corresponding homogenized PDE on a coarse mesh. The two-step procedure described above is, in extreme synthesis, the FE-HMM. The computation of the homogenized coefficient is not trivial, and involves the solution of auxiliary elliptic PDEs, the \textit{corrector equations} (also \textit{cell problems}), on microscopic domains centred at the quadrature points of the macroscopic mesh. In the non-periodic case, there are no closed-form formulas for setting the boundary conditions of the cell problems, as well as the size of the micro domains, which have to be chosen artificially. The resulting mismatch results in the so-called \textit{resonance errors}, which can be dominating in some specific scenarios with respect to numerical and modelling errors. Existing approaches for reducing the resonance error rely on oversampling \cite{YuE07}, on filtering the corrector problems \cite{BlL10}, or on Richardson-extrapolation-type approaches \cite{GlH16}.
 
In this project, we designed two novel corrector problems for determining the homogenized coefficient in the framework of multiscale elliptic PDEs. The first approach consists in solving a parabolic, instead of elliptic, cell problem, and then filtering the resulting correctors with appropriate filtering kernels. A rigorous set-up and analysis of our methodology allowed us to conclude that the resonance error converges in this case with a arbitrarily high order with respect to the size of the micro domains. Fast convergence is achieved through high order filtering kernels, and a trade-off between high order (asymptotic) and computational cost (pre-asymptotic) is demonstrated via thoroughly-designed numerical experiments. We present our results on parabolic corrector problems in the peer-reviewed articles \cite{AAP19,AAP21}. The second methodology is derived from the parabolic approach by considering \textit{modified} elliptic correctors. Indeed, we consider the traditional corrector problems, and modify their right-hand side by adding the solution of our parabolic corrector problems, computed at the final time. Since we only need the solution at final time, we can compute it efficiently by means of appropriate Krylov subspace methods. Again, we rigorously show that our methodology based on modified elliptic problems can be set up to show arbitrarily high order convergence, and demonstrate the accuracy and performances with a series of numerical experiments. Our theoretical and numerical results are published in the peer-reviewed article \cite{AAP19}, and in the research article \cite{AAP20}, currently available as a preprint and submitted for publication. Finally, we adapted the computation of our modified cell problems to a challenging setting of elliptic PDEs with random multiscale coefficients. In this setting, we were able to prove convergence for good proportions of the solution variance, and deduce numerically a conjecture for the unexplained variability. Our results on the stochastic scenario are presented in the research article \cite{AAP20b}, currently available as a preprint and submitted for publication.

\subsection{Local discontinuous Galerkin discretization of elliptic PDEs}\label{sec:Rosilho_1}

\subsection{Stabilized explicit multirate methods for stiff equations}\label{sec:Rosilho_2}

\subsection{Multiscale inverse problems}\label{sec:AndreaDB}

Inverse problems are ubiquitous in diverse fields, such as geology, atmospheric sciences, and medical sciences. A typical example of an inverse problem consists of retrieving information about a medium from partial observation of a physical effect on the medium itself. In mathematical terms, we consider the ill-posed inverse problem of retrieving the data (diffusion coefficient) of an elliptic PDE given discrete noisy observations of its solution. Ill-posedness is circumvented by adding a regularization term to an appropriate minimization problem (e.g., \textit{Tikhonov regularization}), or by reinterpreting all quantities in the problem as random variables and by adopting the Bayesian paradigm. Several works concern inverse problem of this kind, notable references for the Bayesian setting are \cite{CDS10,DaS16,Stu10}.

In this project, we considered the setting of multiscale elliptic PDEs and the problem of retrieving the multiscale diffusion coefficient from noisy discrete observations and given a slow-to-fast-scale parametrization of the diffusion. Our setting is therefore similar to the seminal work \cite{NPS12}. Our goal was to exploit numerical homogenization techniques, such as the FE-HMM, to speed up the solution of the inverse problem, which require otherwise multiple solutions of an expensive multiscale PDE. First, we considered the effects of replacing the full model by its homogenized surrogate in the solution of the inverse problem by Tikhonov regularization. We were able to prove that in the homogenization limit the minimizer solving the inverse problem can be computed by inverting interchangeably the multiscale or the homogenized models, and demonstrated our result with numerical experiments. Our results are published in the peer-reviewed article \cite{AbD19}. We then considered the Bayesian framework, and proved instead convergence of the \textit{posterior measure} that is obtained by replacing the multiscale model with its homogenized surrogate with respect to the Hellinger metrics for probability measures. We demonstrate numerically our theoretical results by considering a challenging Calderon inverse problem, where we aim at reconstructing the full multiscale diffusion coefficient in the whole domain from observations only on the boundary. This setting is especially relevant for applications in electrical impedance tomography. Our results are published in the peer-reviewed article \cite{AbD20}. In the articles \cite{ILS13,ScS17}, the authors design the \textit{ensemble Kalman inversion} (EKI) method, which combines Kalman filtering and Bayesian techniques to compute efficiently the solution of an inverse problem. Inspired by these works, we designed a novel technique based on the EKI and the FE-HMM to solve the inverse problem more efficiently than with standard Monte Carlo techniques. The robustness of homogenization results when combined with the the EKI is rigorously proved, and the EKI is combined with techniques accounting for modelling errors (see \cite{CES14,CDS18}) to solve efficiently Calderon-type inverse problems. Our results are published in the peer-reviewed article \cite{AGZ20}.
 
\subsection{Statistical inference of multiscale diffusion processes}\label{sec:AndreaZ}

Diffusion processes are widely employed to model natural and social phenomena, for example in chemistry, oceanography, atmospheric sciences, and econometrics among others. In some instances, the diffusion process presents two different and widely separated time scales, as for example in chemical reactions where a species reacts much faster (respectively, slower) than all the others. Another example is given by financial markets with a micro/macro structure \cite{AiJ14}. We say in this case that we have a multiscale diffusion process. In all these fields, it is relevant to design techniques to infer the coefficients of a stochastic differential equation (SDE) which models the slow variations of the multiscale diffusion process. In the setting of overdamped Langevin SDEs, such an effective model exists and is given by the theory of homogenization \cite{BLP78,PaS08}. Applying maximum likelihood or Bayesian estimators (see e.g. \cite{PSV09,PSZ13,LiS01,LiS01b}) for the coefficients of the homogenized SDE yields wrong results because the data and the model are incompatible at the fastest time scale \cite{PaS07}. Typically, the biasedness issue is circumvented by subsampling the data \cite{PaS07} or by employing appropriate martingale estimators \cite{KPP15,KKP15}. Subsampling techniques are in particular widely employed \cite{ZMA05} due to their ease of application and versatility. Nevertheless, subsampling is sensitive with respect to the sampling frequency, and can yield extremely biased results.

In this project, we designed a novel methodology based on filtering the multiscale data to infer an effective slow-scale diffusion model. The methodology we proposed has two main advantages with respect to subsampling. First, it does not require knowledge of the \textit{scale separation parameter}, i.e., the ratio between the fast and the slow scales, which is in most applications unknown. Second, it is robust with respect to the tuning parameters of the method, unlike subsampling which is strongly affected by the sampling width. Our results are summarized in the peer-reviewed article \cite{AGP21} and expanded in the research article \cite{GaZ21}, which is available as a preprint and submitted for publication. Our techniques based on filtered data require theoretically continuous data (i.e., high-frequency data) to be applied. In order to treat the case of discrete data, we further developed a robust technique which combines filtering the data and eigenfunction martingale estimating functions (see \cite{KeS99,CrV06,CrV06b,CrV11}) for the same problem. Our results are summarized in the research article \cite{APZ21}, which is available as a preprint and submitted for publication. The research project above led us to find and prove homogenization results for a Poisson problem and an eigenvalue problem defined by the infinitesimal generator of multiscale diffusions of the Langevin kind, as summarized in the research article \cite{Zan21}, which is available as a preprint and submitted for publication. In a slightly different framework, we considered systems of interacting stochastic particles inspired by problems of molecular dynamics, and designed eingfunction martingale estimating functions for inferring their confining and interacting potential. In particular, under the assumption that a single particle can be observed for a long discrete time, we prove that an estimator based on the mean-field limit equation is asymptotically unbiased and efficient. Our results are summarized in the research article \cite{PaZ21}, which is available as a preprint and submitted for publication.

\subsection{Probabilistic numerical methods for forward and inverse problems}\label{sec:Giacomo}

\textit{Probabilistic numerics} (PN) is an emerging field lying at the intersection of numerical analysis, machine learning, informatics, and statistics. The main goal of PN is reinterpreting numerical methods in a probabilistic fashion, and allows for a consistent uncertainty quantification (UQ) of the solution. In particular, if the numerical solution of a computational task is given in the form of a probability measure, instead of a punctual value, it is then possible to propagate uncertainties consistently through pipelines of computations. A notable example is given by inverse problems, for which the replacement of the forward map by a traditional numerical solver can lead to biased and overconfident posterior measures. The principles and literature of PN are summarized in the recent review papers \cite{COS19,HOG15,OaS19}.

In this project, we developed two PN methods for differential equations. First, we considered ODEs and introduced a Runge--Kutta method based on random time steps, the RTS-RK. By perturbing the time step in an appropriate fashion, we introduce a random perturbation on the numerical solution which consistently mimics numerical errors. Moreover, unlike other previous works (e.g. \cite{CGS17}) where PN Runge--Kutta methods are obtained by directly perturbing the solution, randomizing the time steps allows to preserve certain geometric structures of the numerical solvers. In particular, we rigorously studied the numerical conservation of invariants and the near-conservation of the energy for Hamiltonian systems and symplectic integrators. The usefulness of the PN approach is demonstrated via an application of the RTS-RK, combined with the Bayesian approach, to an inverse problem involving ODEs. Our results are summarized in the peer-reviewed article \cite{AbG20}. We then considered PDEs and replicated the idea of perturbing the discretization to build a FEM solver on random meshes, the RM-FEM. As for the RTS-RK, we proved that the RM-FEM introduces probability measures on the FEM solution which are consistent with the numerical errors. More precisely, we proved and demonstrated numerically that it is possible to extract a posteriori error estimators from statistical information carried by the PN solution. Again, we considered an inverse problem involving PDEs and demonstrated the usefulness of the PN approach in this context. Our results are summarized in the research article \cite{AbG21}.

Inspired by the application of PN methods, we considered a theoretical issue arising from the application of approximated and noisy approximations within inverse problems. In this setting, the typical approach relies on pseudo-marginal Monte Carlo methods \cite{AnR09}. A special situation arises when the randomness of the forward map decreases as the map becomes more accurate, as typically happens in PN methods. In this setting, a recent contribution \cite{LST18} showed convergence of appropriate posterior measures to the true posterior. In this work, though, only idealized measures involving inaccessible averages are considered. We expanded the analysis of \cite{LST18} by considering computable measures based on Monte Carlo averages, and computed theoretically the convergence rates with respect to sample size and to the discretization parameter towards the true posterior measures. Our results are summarized in the research article \cite{Gar21b}, currently available as a preprint and submitted for publication.

\section{Research Output}
{\color{red} TO DO}

\bibliographystyle{siamplain}
\bibliography{anmc}

\clearpage
\section{Output Data}

\subsection{Scientific publications}

\subsubsection*{Journal articles}

\sloppy
\emergencystretch=1em
\begin{enumerate}[label={[\arabic*]}]
	\item \bibentry{AAV18}
	\item \bibentry{AAP19}
	\item \bibentry{AAP21}
	\item \bibentry{AbD19}
	\item \bibentry{AbD20}
	\item \bibentry{AbG20}
	\item \bibentry{AbG21}
	\item \bibentry{AGP21}
	\item \bibentry{AGZ20}
	\item \bibentry{APV19}
	\item \bibentry{AbR19}
	\item \bibentry{AbR22}
\end{enumerate}

\subsubsection*{Preprints submitted for publication}

\sloppy
\emergencystretch=1em
\begin{enumerate}[label={[\arabic*]}]
	\item \bibentry{AAP20}
	\item \bibentry{AAP20b}
	\item \bibentry{AGR21}
	\item \bibentry{AGR20}
	\item \bibentry{APZ21}
	\item \bibentry{AbR20c}
	\item \bibentry{CrR21}
	\item \bibentry{Gar21b}
	\item \bibentry{GaZ21}
	\item \bibentry{PaZ21}
	\item \bibentry{Zan21}
\end{enumerate}

\subsubsection*{Technical Reports}

\sloppy
\emergencystretch=1em
\begin{enumerate}[label={[\arabic*]}]
	\item \bibentry{AbR20a}
\end{enumerate}

\subsection{Academic events}

\textbf{Giacomo Garegnani} has given the following presentations on the topics of the project:
\begin{itemize}
	\item \textbf{Workshop presentation:} \textit{Calibration of probabilistic numerical methods}, Dagstuhl Seminar ``Probabilistic Numerical Methods -- From Theory to Implementation'', October 2021, Dagstuhl, Germany
	\item \textbf{Conference presentation:} \textit{Random mesh FEM: A probabilistic approach to the FEM}, European Finite Element Fair, September 2021, Paris, France
	\item \textbf{Seminar presentation:} \textit{Filtering the data: An alternative to subsampling for drift estimation of multiscale diffusions}, RWTH Aachen University, January 2021, Aachen, Germany
	\item \textbf{Conference presentation (cancelled due to Covid19):} \textit{Model misspecification and uncertainty quantification for drift estimation in multiscale diffusion processes}, SIAM conference on uncertainty quantification, March 2020, Garching, Germany
	\item \textbf{Seminar presentation:} \textit{A pre-processing technique for asymptotically correct drift estimation in multiscale diffusion processes}, Imperial College London, February 2020, London, UK
	\item \textbf{Seminar presentation:} \textit{Bayesian inference of multiscale differential equations}, Caltech, August 2019, Pasadena, US
	\item \textbf{Seminar presentation:} \textit{Bayesian inference of multiscale diffusion processes}, MATHICSE retreat, June 2019, Champéry, Switzerland
	\item \textbf{Summer school presentation:} \textit{Probabilistic Runge--Kutta methods	for uncertainty quantification of numerical errors in geometric integration}, FoMICS-DADSi Summer School on Data Assimilation, September 2018,  Lugano, Switzerland
	\item \textbf{Conference presentation:} \textit{Uncertainty quantification of numerical errors in geometric integration via random time steps}, AIMS Conference on Dynamical Systems, Differential Equations and Applications, July 2018, Taipei, Taiwan
	\item \textbf{Seminar presentation:} \textit{Probabilistic geometric integration of ordinary differential equations}, MATHICSE retreat, June 2018, Sainte-Croix, Switzerland
	\item \textbf{Conference presentation:} \textit{Random time steps geometric integrators of ordinary differential equations for uncertainty quantification of numerical errors}, Swiss Numerics Day, April 2018, ETH Zürich, Switzerland
	\item \textbf{Seminar presentation:} \textit{Uncertainty quantification of numerical errors in geometric integration via random time steps}, Max Planck Institute for Intelligent Systems, March 2018, Tübingen, Germany 
	\item \textbf{Seminar presentation:} \textit{Probabilistic Runge--Kutta methods for ODEs: Chaotic problems and geometric properties}, MATHICSE retreat, June 2017, Leysin, Switzerland
\end{itemize}
%
\textbf{Andrea Zanoni} has given the following presentations on the topics of the project:
\begin{itemize}
	\item \textbf{Conference presentation (upcoming):} \textit{Data-driven homogenization of multiscale Langevin dynamics}, SIAM Conference on uncertainty quantification, April 2022, Atlanta, US
	\item \textbf{Conference presentation:} \textit{Inference of effective diffusions from multiscale data}, Swiss Numerics Day, September 2021, Lausanne, Switzerland
	\item \textbf{Conference presentation:} \textit{Solution of multiscale inverse problems through filtering techniques and numerical homogenization}, 14th WCCM \& ECCOMAS Congress 2020, January 2021, virtual event
\end{itemize}
%
\textbf{Andrea Di Blasio} has given the following presentations on the topics of the project:
\begin{itemize}
	\item \textbf{Conference presentation:} \textit{Model order reduction and numerical homogenization for solving Bayesian multiscale inverse problems}, EUROMECH colloquium, August 2018, Bad-Herrenalb, Germany
	\item \textbf{Seminar presentation:} \textit{Numerical methods for solving Bayesian multiscale inverse problems}, MATHICSE retreat, June 2018, Sainte-Croix, Switzerland
	\item \textbf{Conference presentation:} \textit{Numerical homogenization and model order reduction for solving linear elasticity problems in perforated domains}, ECCM -- ECFD 2018, June 2018, Glasgow, UK
	\item \textbf{Conference presentation:} \textit{Model order reduction and Bayesian techniques for multiscale inverse problems}, COMPLAS 2017, September 2017, Barcelona, Spain
	\item \textbf{Conference presentation:} \textit{Numerical homogenization and Bayesian techniques for multiscale inverse problems}, EQUADIFF 2017, July 2017, Bratislava, Slovakia
	\item \textbf{Seminar presentation:} \textit{Using numerical homogenization for solving elliptic multiscale inverse problems}, MATHICSE retreat, June 2017, Leysin, Switzerland
	\item \textbf{Conference presentation:} \textit{Solving elliptic multiscale inverse problems using Bayesian techniques and numerical homogenization}, Swiss Numerics Day, April 2017, Basel, Switzerland
	\item \textbf{Seminar presentation:} \textit{A reduced basis method for multiscale inverse problems}, MATHICSE retreat, June 2016, Leysin, Switzerland
	\item \textbf{Conference poster:} \textit{A reduced basis method for multiscale inverse problems}, Swiss Numerics Day, April 2016, Fribourg, Switzerland
\end{itemize}
%
\textbf{Giacomo Rosilho de Souza} has given the following presentations on the topics of the project:
\begin{itemize}
	\item \textbf{Conference presentation:} \textit{Multirate stabilized explicit methods based on a modified equation for problems with multiple scales}, SIAM conference on computational science and engineering (CSE21), March 2021, virtual event
	\item \textbf{Workshop presentation:} \textit{Multirate stabilized explicit methods for deterministic and stochastic differential equations without clear-cut scale separation}, CECAM workshop ``Multiscale simulations of soft matter: New method developments and mathematical foundations'', September 2020, Mainz, Germany, virtual event 
	\item \textbf{Seminar presentation:} \textit{Multirate explicit stabilized integrators for stiff differential equations}, Universität Basel, November 2019, Basel, Switzerland
	\item \textbf{Conference presentation:} \textit{Stabilized explicit multirate methods for ordinary and stochastic differential equations with multiple scales}, SciCADE International Conference on Scientific Computation and Differential Equations, July 2019, Innsbruck, Austria
	\item \textbf{Seminar presentation:} \textit{Multirate explicit stabilized integrators for stiff differential equations}, MATHICSE seminar, June 2019, Champéry, Switzerland
	\item \textbf{Seminar presentation:} \textit{A local discontinuous Galerkin FEM for linear and quasilinear elliptic equations}, MATHICSE seminar, June 2018, Sainte-Croix, Switzerland
	\item \textbf{Conference presentation:} \textit{A priori and a posteriori analysis of a local scheme for elliptic equations}, Swiss Numerics Day, April 2018, Zürich, Switzerland
	\item \textbf{Seminar presentation:} \textit{Predictor corrector local time stepping scheme for parabolic equations}, MATHICSE seminar, June 2017, Leysin, Switzerland
	\item \textbf{Seminar presentation:} \textit{Two local time stepping techniques for parabolic equations}, MATHICSE seminar, June 2016, Leysin, Switzerland
\end{itemize}
%
\textbf{Assyr Abdulle} has given the following presentations on the topics of the project:
\begin{itemize}
	\item \textbf{Workshop presentation:} \textit{Learning effective models from multiscale data: filtering and	Bayesian inference}, Oberwolfach workshop ``Geometric Numerical Integration'', March 2021, Oberwolfach, Germany
	\item \textbf{Workshop presentation:} \textit{Stabilized explicit multirate methods for ordinary and stochastic differential equations with multiple scales}, Workshop on multiscale methods for deterministic and stochastic dynamics, January 2020, Geneva, Switzerland
	\item \textbf{Plenary conference presentation:} \textit{Numerical methods for wave propagation in heterogenenous media}, 14th International Conference on Mathematical and Numerical Aspects of Wave Propagation (WAVES 2019), August 2019, Vienna, Austria
	\item \textbf{Workshop presentation:} \textit{A Bayesian approach for multiscale inverse problems}, Banff International Research Station workshop ``Integrating the integrators for nonlinear evolution equations: from analysis to numerical methods, high-performance computing and applications'', December 2018, Banff, Canada
	\item \textbf{Workshop presentation:} \textit{Bayesian numerical homogenization methods for multiscale inverse problems}, Workshop on Numerical methods for multiscale PDEs, September 2018, Cargese, France
	\item \textbf{Workshop presentation:} \textit{Bayesian multiscale inverse problems and probabilistic numerical methods}, Workshop on Interplay of multiscale data assimilation and data science with advanced PDE discretizations, Erwin Schrödinger International Institute for Mathematics and Physics (ESI), June 2018, Vienna, Austria
	\item \textbf{Workshop presentation:} \textit{Probabilistic numerical methods and Bayesian multiscale inverse problems}, Workshop on Data driven modelling of complex systems, Alan Turing Institute, May 2018, London, UK
\end{itemize}

\subsection{Knowledge transfer events}

\textbf{Giacomo Garegnani} participated in the following knowledge transfer events:
\begin{itemize}
	\item \textbf{Summer school:} Dobbiaco summer school on Probabilistic Numerics, June 2017, Dobbiaco, Italy
	\item \textbf{Workshop:} Workshop on probabilistic numerical methods, Alan Turing Institute, April 2018, London, UK
	\item \textbf{Co-Supervision of Master Projects:} 
	\begin{itemize}
		\item Andrea Zanoni, \textit{Ensemble Kalman filter for multiscale inverse problems}, EPFL, 2019 -- co-supervised with Assyr Abdulle and Sandro Salsa (Politecnico di Milano)
		\item Aleksa Stanković, \textit{Probabilistic methods for differential equations: adaptivity and Bayesian inverse problems}, EPFL, 2018 -- co-supervised with Assyr Abdulle
	\end{itemize}
	\item \textbf{Co-Supervision of Semester Projects:}
	\begin{itemize}
		\item Daniele Hamm, \textit{Numerical study of an iterative filtering method for drift estimation of multiscale diffusions}, EPFL, 2021 -- co-supervised with Assyr Abdulle
		\item Anne-Sophie Van De Velde, \textit{Parameter estimation in multiscale Langevin dynamics with particle filters and Monte Carlo methods}, EPFL, 2020 -- co-supervised with Assyr Abdulle
		\item Wojciech Reise, \textit{Probabilistic solvers for ordinary differential equations}, EPFL, 2019 -- co-supervised with Assyr Abdulle
	\end{itemize}
\end{itemize}
%
\textbf{Giacomo Rosilho de Souza} participated in the following knowledge transfer events:
\begin{itemize}
	\item \textbf{Co-Supervision of Master Projects:}
	\begin{itemize}
		\item Lia Gander, \textit{Optimized Chebyshev methods for discrete stochastic simulations}, EPFL, 2019 -- co-supervised with Assyr Abdulle
		\item Tristan Chanay, \textit{Optimal explicit stabilized method for jump-diffusion processes}, EPFL, 2021 -- co-supervised with Assyr Abdulle
	\end{itemize} 
\end{itemize}

\subsection{Collaboration}

The following national and international collaborations took place:
\begin{itemize}
	\item Prof. Grigorios A. Pavliotis, Department of Mathematics, Imperial College London, London, UK; including: 
	\begin{itemize}
		\item scientific visit of Prof. Grigorios A. Pavliotis at EPFL (March 2019)
		\item scientific visit of Giacomo Garegnani at Imperial College London (3.2.2020-7.2.2020)
		\item planned scientific visit of Andrea Zanoni at Imperial College London (14.02.2022-18.03.2022)
	\end{itemize}
	\item Prof. Andrew M. Stuart, Department of Computing and Mathematical Sciences, Caltech, Pasadena, US; including a scientific visit of Giacomo Garegnani at Caltech (19.08.2019-20-09-2019)
	\item Prof. Marcus J. Grote, Departement Mathematik und Informatik, Universität Basel, Basel, Switzerland; including a seminar by Giacomo Rosilho de Souza at Unversität Basel
	\item MER Gilles Vilmart, Section de Mathématiques, Université de Geneve, Geneva, Switzerland
\end{itemize}

\subsection{Awards}

\begin{itemize}
	\item Giacomo Rosilho de Souza won the John Butcher prize in numerical analysis for the talk \textit{Multirate explicit stabilized integrators for stiff differential equations} that he gave at the SciCADE conference, held in July 2019, Innsbruck, Austria
	\item Giacomo Garegnani won a SIAM travel award to join the SIAM conference on uncertainty quantification, held in March 2020, Garching, Germany (cancelled due to Covid19) 
\end{itemize}

\end{document}  
