\documentclass[10pt]{article}

% Packages and macros
% Common packages
\usepackage[T1]{fontenc}
\usepackage{lmodern}
\usepackage[utf8]{inputenc}
\usepackage{vmargin}
\usepackage{mathrsfs, mathenv}
\usepackage{amsmath, amsthm, amssymb, amsfonts, amscd}
\usepackage{graphicx}
\usepackage{hyperref}
\hypersetup{citecolor=blue, colorlinks=true, linkcolor=black}
\usepackage[capitalise]{cleveref}
\usepackage{autonum}

% bibliography
\usepackage{bibentry}
\nobibliography*

% tables
\usepackage{booktabs}

% plots
\usepackage{pgfplots}
\usepackage{tikz}
\usetikzlibrary{patterns,arrows,decorations.pathmorphing,backgrounds,positioning,fit,matrix}
\usepackage[labelfont=bf]{caption}
\setlength{\belowcaptionskip}{-5pt}
\usepackage{here}
\usepackage[font=normal]{subcaption}

% Prevent itemized lists from running into the left margin inside theorems and proofs
\usepackage{enumitem}
\setlist[itemize]{leftmargin=.5in}
\setlist[enumerate]{leftmargin=.5in,topsep=3pt,itemsep=3pt,label=(\roman*)}

\newcommand{\abs}[1]{\left\lvert#1\right\rvert}
\newcommand{\norm}[1]{\left\|#1\right\|}
\renewcommand{\phi}{\varphi}
\renewcommand{\theta}{\vartheta}
\newcommand{\totext}[1]{\ensuremath{\stackrel{#1}{\to}}}
\newcommand{\R}{\mathbb{R}}
\newcommand{\epl}{\varepsilon}
\newcommand{\defeq}{\coloneqq}
\newcommand{\eqdef}{\eqqcolon}
\newcommand{\E}{\operatorname{\mathbb{E}}}
\newcommand{\Hell}{d_{\mathrm{Hell}}}
\renewcommand{\d}{\mathrm{d}}
\newcommand{\dd}{\,\mathrm{d}}


% Title
\title{Scientific Report on SNSF project 172710 \\ Numerical methods for stationary and time-dependent multiscale problems}
\author{Professor Assyr Abdulle \\
	Chair of Numerical Analysis and Computational Mathematics (ANMC) \\
	MATH, EPFL, Station 8, 1015 Lausanne, Switzerland
}
\date{}

\begin{document}
	
\maketitle	

\section{Summary of the research work and its results}

In this project we have focused on a series of diverse aspects of multiscale differential problems \cite{BLP78,CiD99} and their numerical solution \cite{AEE12}. Multiscale differential equations arise in a wide range of models and find applications in both applied and social sciences. Often, the discretization needed for solving multiscale equations numerically is constrained by the characteristic length of the smallest size in the problem. For this reason, we employ the theory of homogenization, which allows to obtain effective models which give a macroscopic description of the multiscale solution, and in practice to solve numerically the equations. The multiplicity of scales in the problem could be spatial, e.g. in elasticity or heat transfer problems for heterogeneous materials, or temporal, e.g. in modelling chemical reactions with a fast/slow behaviour or in modelling financial markets with a micro/macro structure.

The content of this project can be roughly divided in two parts. The first concerns the numerical solution of multiscale \textit{forward} problems, and is summarized in \cref{sec:Edoardo,sec:Rosilho_1,sec:Rosilho_2}. In this part, we consider the data of the problem (i.e., elasticity of a heterogeneous material, volatility of a financial market, ...) to be given, and design novel analytical and numerical techniques for approximate the solution of the underlying differential equations. In the second part, which is summarized in \cref{sec:AndreaDB,sec:AndreaZ} we instead consider multiscale \textit{inverse} problems, for which the data is the unknown, and a solution is given as observations. The more fundamental topic of \textit{probabilistic numerical methods}, which can be employed to milden some issues in both the forward and the inverse solutions, is finally summarized in \cref{sec:Giacomo}.

\subsection{Corrector problems with enhanced resonance errors in the FE-HMM}\label{sec:Edoardo}

The finite element heterogeneous multiscale method (FE-HMM) \cite{Abd05b,EE03} is a numerical scheme designed for partial differential equations (PDEs) with highly-oscillatory coefficients. In this setting, a fine mesh is needed to approximate the solution with a standard finite element solver. Indeed, the mesh characteristic size cannot exceed the oscillation size if the goal is capturing the microscopic features of the solution. If we are only interested in the macroscopic behaviour, we can instead rely on to the theory of homogenization \cite{BLP78,CiD99} and first compute an effective, or \textit{homogenized}, non-oscillatory coefficient, and then solve numerically the corresponding homogenized PDE on a coarse mesh. The two-step procedure described above is, in extreme synthesis, the FE-HMM. The computation of the homogenized coefficient is not trivial, and involves the solution of auxiliary elliptic PDEs, the \textit{corrector equations} (also \textit{cell problems}), on microscopic domains centred at the quadrature points of the macroscopic mesh. In the non-periodic case, there are no closed-form formulas for setting the boundary conditions of the cell problems, as well as the size of the micro domains, which have to be chosen artificially. The resulting mismatch results in the so-called \textit{resonance errors}, which can be dominating in some specific scenarios with respect to numerical and modelling errors. Existing approaches for reducing the resonance error rely on oversampling \cite{YuE07}, on filtering the corrector problems \cite{BlL10}, or on Richardson-extrapolation-type approaches \cite{GlH16}.
 
In this project, we designed two novel corrector problems for determining the homogenized coefficient in the framework of multiscale elliptic PDEs. The first approach consists in solving a parabolic, instead of elliptic, cell problem, and then filtering the resulting correctors with appropriate filtering kernels. A rigorous set-up and analysis of our methodology allowed us to conclude that the resonance error converges in this case with a arbitrarily high order with respect to the size of the micro domains. Fast convergence is achieved through high order filtering kernels, and a trade-off between high order (asymptotic) and computational cost (pre-asymptotic) is demonstrated via thoroughly-designed numerical experiments. We present our results on parabolic corrector problems in the peer-reviewed articles \cite{AAP19,AAP21}. The second methodology is derived from the parabolic approach by considering \textit{modified} elliptic correctors. Indeed, we consider the traditional corrector problems, and modify their right-hand side by introducing time averages of the solution of our parabolic corrector problems. We describe how to perform the computation of these time averages efficiently by means of appropriate Krylov subspace methods. Again, we rigorously show that our methodology based on modified elliptic problems can be set up to show arbitrarily high order convergence, and demonstrate the accuracy and performances with a series of numerical experiments. Our theoretical and numerical results are published in the peer-reviewed article \cite{AAP19}, and in the research article \cite{AAP20}, currently available as a preprint and submitted for publication. Finally, we adapted the computation of our modified cell problems to a challenging setting of elliptic PDEs with random multiscale coefficients. In this setting, we were able to prove convergence for good proportions of the solution variance, and deduce numerically a conjecture for the unexplained variability. Our results on the stochastic scenario are presented in the research article \cite{AAP20b}, currently available as a preprint and submitted for publication.

\subsection{Local discontinuous Galerkin discretization of elliptic PDEs}\label{sec:Rosilho_1}

\subsection{Stabilized explicit multirate methods for stiff equations}\label{sec:Rosilho_2}

\subsection{Multiscale inverse problems}\label{sec:AndreaDB}

\subsection{Statistical inference of multiscale SDEs}\label{sec:AndreaZ}

\subsection{Probabilistic numerical methods for forward and inverse problems}\label{sec:Giacomo}


\section{Research Output}
{\color{red} TO DO}

\bibliographystyle{siamplain}
\bibliography{anmc}

\clearpage
\section{Output Data}

\subsection{Scientific publications}

\subsubsection*{Journal articles}

\sloppy
\emergencystretch=1em
\begin{enumerate}[label={[\arabic*]}]
	\item \bibentry{AAV18}
	\item \bibentry{AAP19}
	\item \bibentry{AAP21}
	\item \bibentry{AbD19}
	\item \bibentry{AbD20}
	\item \bibentry{AbG20}
	\item \bibentry{AbG21}
	\item \bibentry{AGP21}
	\item \bibentry{AGZ20}
	\item \bibentry{APV19}
	\item \bibentry{AbR19}
\end{enumerate}

\subsubsection*{Preprints submitted for publication}

\sloppy
\emergencystretch=1em
\begin{enumerate}[label={[\arabic*]}]
	\item \bibentry{AAP20}
	\item \bibentry{AAP20b}
	\item \bibentry{AGR21}
	\item \bibentry{AGR20}
	\item \bibentry{APZ21}
	\item \bibentry{AbR20b}
	\item \bibentry{AbR20c}
	\item \bibentry{CrR21}
	\item \bibentry{Gar21b}
	\item \bibentry{GaZ21}
	\item \bibentry{PaZ21}
	\item \bibentry{Zan21}
\end{enumerate}

\subsubsection*{Technical Reports}

\sloppy
\emergencystretch=1em
\begin{enumerate}[label={[\arabic*]}]
	\item \bibentry{AbR20a}
\end{enumerate}

\subsection{Academic events}

\textbf{Giacomo Garegnani} has given the following presentations on the topics of the project:
\begin{itemize}
	\item \textbf{Workshop presentation:} \textit{Calibration of probabilistic numerical methods}, Dagstuhl Seminar ``Probabilistic Numerical Methods -- From Theory to Implementation'', October 2021, Dagstuhl, Germany
	\item \textbf{Conference presentation:} \textit{Random mesh FEM: A probabilistic approach to the FEM}, European Finite Element Fair, September 2021, Paris, France
	\item \textbf{Seminar presentation:} \textit{Filtering the data: An alternative to subsampling for drift estimation of multiscale diffusions}, RWTH Aachen University, January 2021, Aachen, Germany
	\item \textbf{Conference presentation (cancelled due to Covid19):} \textit{Model misspecification and uncertainty quantification for drift estimation in multiscale diffusion processes}, SIAM conference on uncertainty quantification, March 2020, Garching, Germany
	\item \textbf{Seminar presentation:} \textit{A pre-processing technique for asymptotically correct drift estimation in multiscale diffusion processes}, Imperial College London, February 2020, London, UK
	\item \textbf{Seminar presentation:} \textit{Bayesian inference of multiscale differential equations}, Caltech, August 2019, Pasadena, US
	\item \textbf{Seminar presentation:} \textit{Bayesian inference of multiscale diffusion processes}, MATHICSE retreat, June 2019, Champéry, Switzerland
	\item \textbf{Summer school presentation:} \textit{Probabilistic Runge--Kutta methods	for uncertainty quantification of numerical errors in geometric integration}, FoMICS-DADSi Summer School on Data Assimilation, September 2018,  Lugano, Switzerland
	\item \textbf{Conference presentation:} \textit{Uncertainty quantification of numerical errors in geometric integration via random time steps}, AIMS Conference on Dynamical Systems, Differential Equations and Applications, July 2018, Taipei, Taiwan
	\item \textbf{Seminar presentation:} \textit{Probabilistic geometric integration of ordinary differential equations}, MATHICSE retreat, June 2018, Sainte-Croix, Switzerland
	\item \textbf{Conference presentation:} \textit{Random time steps geometric integrators of ordinary differential equations for uncertainty quantification of numerical errors}, Swiss Numerics Day, April 2018, ETH Zürich, Switzerland
	\item \textbf{Seminar presentation:} \textit{Uncertainty quantification of numerical errors in geometric integration via random time steps}, Max Planck Institute for Intelligent Systems, March 2018, Tübingen, Germany 
	\item \textbf{Seminar presentation:} \textit{Probabilistic Runge--Kutta methods for ODEs: Chaotic problems and geometric properties}, MATHICSE retreat, June 2017, Leysin, Switzerland
\end{itemize}
%
\textbf{Andrea Zanoni} has given the following presentations on the topics of the project:
\begin{itemize}
	\item \textbf{Conference presentation (upcoming):} \textit{Data-driven homogenization of multiscale Langevin dynamics}, SIAM Conference on uncertainty quantification, April 2022, Atlanta, US
	\item \textbf{Conference presentation:} \textit{Inference of effective diffusions from multiscale data}, Swiss Numerics Day, September 2021, Lausanne, Switzerland
	\item \textbf{Conference presentation:} \textit{Solution of multiscale inverse problems through filtering techniques and numerical homogenization}, 14th WCCM \& ECCOMAS Congress 2020, January 2021, virtual event
\end{itemize}
%
\textbf{Andrea Di Blasio} has given the following presentations on the topics of the project:
\begin{itemize}
	\item \textbf{Conference presentation:} \textit{Model order reduction and numerical homogenization for solving Bayesian multiscale inverse problems}, EUROMECH colloquium, August 2018, Bad-Herrenalb, Germany
	\item \textbf{Seminar presentation:} \textit{Numerical methods for solving Bayesian multiscale inverse problems}, MATHICSE retreat, June 2018, Sainte-Croix, Switzerland
	\item \textbf{Conference presentation:} \textit{Numerical homogenization and model order reduction for solving linear elasticity problems in perforated domains}, ECCM -- ECFD 2018, June 2018, Glasgow, UK
	\item \textbf{Conference presentation:} \textit{Model order reduction and Bayesian techniques for multiscale inverse problems}, COMPLAS 2017, September 2017, Barcelona, Spain
	\item \textbf{Conference presentation:} \textit{Numerical homogenization and Bayesian techniques for multiscale inverse problems}, EQUADIFF 2017, July 2017, Bratislava, Slovakia
	\item \textbf{Seminar presentation:} \textit{Using numerical homogenization for solving elliptic multiscale inverse problems}, MATHICSE retreat, June 2017, Leysin, Switzerland
	\item \textbf{Conference presentation:} \textit{Solving elliptic multiscale inverse problems using Bayesian techniques and numerical homogenization}, Swiss Numerics Day, April 2017, Basel, Switzerland
	\item \textbf{Seminar presentation:} \textit{A reduced basis method for multiscale inverse problems}, MATHICSE retreat, June 2016, Leysin, Switzerland
	\item \textbf{Conference poster:} \textit{A reduced basis method for multiscale inverse problems}, Swiss Numerics Day, April 2016, Fribourg, Switzerland
\end{itemize}
%
\textbf{Giacomo Rosilho de Souza} has given the following presentations on the topics of the project:
\begin{itemize}
	\item \textbf{Conference presentation:} \textit{Multirate stabilized explicit methods based on a modified equation for problems with multiple scales}, SIAM conference on computational science and engineering (CSE21), March 2021, virtual event
	\item \textbf{Workshop presentation:} \textit{Multirate stabilized explicit methods for deterministic and stochastic differential equations without clear-cut scale separation}, CECAM workshop ``Multiscale simulations of soft matter: New method developments and mathematical foundations'', September 2020, Mainz, Germany, virtual event 
	\item \textbf{Seminar presentation:} \textit{Multirate explicit stabilized integrators for stiff differential equations}, Universität Basel, November 2019, Basel, Switzerland
	\item \textbf{Conference presentation:} \textit{Stabilized explicit multirate methods for ordinary and stochastic differential equations with multiple scales}, SciCADE International Conference on Scientific Computation and Differential Equations, July 2019, Innsbruck, Austria
	\item \textbf{Seminar presentation:} \textit{Multirate explicit stabilized integrators for stiff differential equations}, MATHICSE seminar, June 2019, Champéry, Switzerland
	\item \textbf{Seminar presentation:} \textit{A local discontinuous Galerkin FEM for linear and quasilinear elliptic equations}, MATHICSE seminar, June 2018, Sainte-Croix, Switzerland
	\item \textbf{Conference presentation:} \textit{A priori and a posteriori analysis of a local scheme for elliptic equations}, Swiss Numerics Day, April 2018, Zürich, Switzerland
	\item \textbf{Seminar presentation:} \textit{Predictor corrector local time stepping scheme for parabolic equations}, MATHICSE seminar, June 2017, Leysin, Switzerland
	\item \textbf{Seminar presentation:} \textit{Two local time stepping techniques for parabolic equations}, MATHICSE seminar, June 2016, Leysin, Switzerland
\end{itemize}
%
\textbf{Assyr Abdulle} has given the following presentations on the topics of the project:
\begin{itemize}
	\item \textbf{Workshop presentation:} \textit{Learning effective models from multiscale data: filtering and	Bayesian inference}, Oberwolfach workshop ``Geometric Numerical Integration'', March 2021, Oberwolfach, Germany
	\item \textbf{Workshop presentation:} \textit{Stabilized explicit multirate methods for ordinary and stochastic differential equations with multiple scales}, Workshop on multiscale methods for deterministic and stochastic dynamics, January 2020, Geneva, Switzerland
	\item \textbf{Plenary conference presentation:} \textit{Numerical methods for wave propagation in heterogenenous media}, 14th International Conference on Mathematical and Numerical Aspects of Wave Propagation (WAVES 2019), August 2019, Vienna, Austria
	\item \textbf{Workshop presentation:} \textit{A Bayesian approach for multiscale inverse problems}, Banff International Research Station workshop ``Integrating the integrators for nonlinear evolution equations: from analysis to numerical methods, high-performance computing and applications'', December 2018, Banff, Canada
	\item \textbf{Workshop presentation:} \textit{Bayesian numerical homogenization methods for multiscale inverse problems}, Workshop on Numerical methods for multiscale PDEs, September 2018, Cargese, France
	\item \textbf{Workshop presentation:} \textit{Bayesian multiscale inverse problems and probabilistic numerical methods}, Workshop on Interplay of multiscale data assimilation and data science with advanced PDE discretizations, Erwin Schrödinger International Institute for Mathematics and Physics (ESI), June 2018, Vienna, Austria
	\item \textbf{Workshop presentation:} \textit{Probabilistic numerical methods and Bayesian multiscale inverse problems}, Workshop on Data driven modelling of complex systems, Alan Turing Institute, May 2018, London, UK
\end{itemize}

\subsection{Knowledge transfer events}

\textbf{Giacomo Garegnani} participated in the following knowledge transfer events:
\begin{itemize}
	\item \textbf{Summer school:} Dobbiaco summer school on Probabilistic Numerics, June 2017, Dobbiaco, Italy
	\item \textbf{Workshop:} Workshop on probabilistic numerical methods, Alan Turing Institute, April 2018, London, UK
	\item \textbf{Co-Supervision of Master Projects:} 
	\begin{itemize}
		\item Andrea Zanoni, \textit{Ensemble Kalman filter for multiscale inverse problems}, EPFL, 2019 -- co-supervised with Assyr Abdulle and Sandro Salsa (Politecnico di Milano)
		\item Aleksa Stanković, \textit{Probabilistic methods for differential equations: adaptivity and Bayesian inverse problems}, EPFL, 2018 -- co-supervised with Assyr Abdulle
	\end{itemize}
	\item \textbf{Co-Supervision of Semester Projects:}
	\begin{itemize}
		\item Daniele Hamm, \textit{Numerical study of an iterative filtering method for drift estimation of multiscale diffusions}, EPFL, 2021 -- co-supervised with Assyr Abdulle
		\item Anne-Sophie Van De Velde, \textit{Parameter estimation in multiscale Langevin dynamics with particle filters and Monte Carlo methods}, EPFL, 2020 -- co-supervised with Assyr Abdulle
		\item Wojciech Reise, \textit{Probabilistic solvers for ordinary differential equations}, EPFL, 2019 -- co-supervised with Assyr Abdulle
	\end{itemize}
\end{itemize}
%
\textbf{Giacomo Rosilho de Souza} participated in the following knowledge transfer events:
\begin{itemize}
	\item \textbf{Co-Supervision of Master Projects:}
	\begin{itemize}
		\item Lia Gander, \textit{Optimized Chebyshev methods for discrete stochastic simulations}, EPFL, 2019 -- co-supervised with Assyr Abdulle
		\item Tristan Chanay, \textit{Optimal explicit stabilized method for jump-diffusion processes}, EPFL, 2021 -- co-supervised with Assyr Abdulle
	\end{itemize} 
\end{itemize}

\subsection{Collaboration}

The following national and international collaborations took place:
\begin{itemize}
	\item Prof. Grigorios A. Pavliotis, Department of Mathematics, Imperial College London, London, UK; including: 
	\begin{itemize}
		\item scientific visit of Prof. Grigorios A. Pavliotis at EPFL (March 2019)
		\item scientific visit of Giacomo Garegnani at Imperial College London (3.2.2020-7.2.2020)
		\item planned scientific visit of Andrea Zanoni at Imperial College London (14.02.2022-18.03.2022)
	\end{itemize}
	\item Prof. Andrew M. Stuart, Department of Computing and Mathematical Sciences, Caltech, Pasadena, US; including a scientific visit of Giacomo Garegnani at Caltech (19.08.2019-20-09-2019)
	\item Prof. Marcus J. Grote, Departement Mathematik und Informatik, Universität Basel, Basel, Switzerland
	\item MER Gilles Vilmart, Section de Mathématiques, Université de Geneve, Geneva, Switzerland
\end{itemize}

\subsection{Awards}

\begin{itemize}
	\item Giacomo Rosilho de Souza won the John Butcher prize in numerical analysis for the talk \textit{Multirate explicit stabilized integrators for stiff differential equations} that he gave at the SciCADE conference, held in July 2019, Innsbruck, Austria
	\item Giacomo Garegnani won a SIAM travel award to join the SIAM conference on uncertainty quantification, held in March 2020, Garching, Germany (cancelled due to Covid19) 
\end{itemize}

\end{document}  
